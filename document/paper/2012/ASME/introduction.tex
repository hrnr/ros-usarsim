\section*{INTRODUCTION}

Robotic simulation is very important in developing robotics applications, both
for rapid prototyping of applications, behaviors, scenarios, and for debugging
purposes of many high-level tasks.

Simulation environments enable researchers to focus on the algorithm development without having to worry about the hardware aspects of the robots. If correctly implemented, simulation can be an effective first step in the development and deployment of new algorithms. Simulation provides extensive testing opportunities without risking of harm to personnel or equipment. Major components of the robotic architecture (for example, advanced sensors) can be simulated and enable the developers to focus on the algorithms or components in which they are interested. This can be useful when development teams are working in parallel or when experimenting with novel technological components that may not be fully implemented yet.

Simulation can be used to provide access to environments that would normally not be available to the development team. Particular test scenarios can be run repeatedly, with the assurance that conditions are identical each time. The environmental conditions, such as time of day, lighting, or weather conditions, as well as the position and behavior of other entities in the world can be fully controlled. In terms of performance evaluation, it can truly provide an ``apples-to-apples" comparison of different software running on identical hardware platforms in identical environments. Another important feature of a robotic simulator is easy integration of different robotic platforms, different scenarios, different objects in the scene, as well
as support for multi-robot applications.

The different aspects mentioned above are handled by different flavors of simulator~\cite{ZARATTI.LNAI.2007}.
The basic premise of this paper is to describe the interface connecting the Unified System for Automation and Robot Simulation (USARSim) framework with the Robot Operating System (ROS) framework for manufacturing applications. The following sections describe, analyze and illustrate the new interface for the navigation of mobile robot and robotic arm.

%The rest of this paper is organized as follows: Section~\ref{s:background} gives an overview of USARSim and ROS. Section~\ref{s:topics} describes the specification of the topics addressed in ROS. Section~\ref{s:interface} details the interface and the associated commands to setup and run the interface. Section~\ref{s:navigation} displays examples of the interface running the navigation and motion stacks for mobile robots and robotic arms, respectively. Section~\ref{s:conclusion} concludes this paper and gives an overview of the future work. 